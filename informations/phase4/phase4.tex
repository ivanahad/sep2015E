
\documentclass[a4paper, 12pt]{article}

\usepackage[french]{babel} 
\usepackage[utf8]{inputenc}
\usepackage[T1]{fontenc} 
\usepackage{amsmath}
\usepackage[toc,page]{appendix}
\usepackage{amssymb}
\usepackage{listings}  
\usepackage{graphicx}
\usepackage[margin=2.5cm]{geometry}
\usepackage{amsmath,amsfonts,amssymb}
\usepackage{hyperref}
\lstset{
language=Java,
breaklines=true
}



\newcommand*{\plogo}{\fbox{$\mathcal{PL}$}} % Generic publisher logo

%----------------------------------------------------------------------------------------
%	TITLE PAGE
%----------------------------------------------------------------------------------------

\newcommand*{\titleGM}{\begingroup % Create the command for including the title page in the document
\hbox{ % Horizontal box
\hspace*{0.2\textwidth} % Whitespace to the left of the title page
\rule{2pt}{\textheight} % Vertical line
\hspace*{0.05\textwidth} % Whitespace between the vertical line and title page text
\parbox[b]{0.75\textwidth}{ % Paragraph box which restricts text to less than the width of the page

{\noindent\Huge\bfseries Software Project\\ Engineering }\\[2\baselineskip] % Title
{\Large \textit{Phase 4 Report}}\\[4\baselineskip] % Tagline or further description
{\Large \textbf{Project leader} : Hélène Verhaeghe}
\\
{\Large \textsc{\textbf{Group E}}\\\textsc{Aurian De Potter(Group leader)},\\ \textsc{Eddy Ndizera},\\ \textsc{Ivan Ahad},\\ \textsc{Arnaud Dethise},\\ \textsc{Ludovic Fastré},\\ \textsc{Anthony Dechamps},\\ \textsc{Geoffroy Husson},\\ \textsc{Jonathan Legat}} % Author name

\vspace{0.5\textheight} % Whitespace between the title block and the publisher
{\noindent \Large \textbf{INGI2255}}\\[\baselineskip] % Publisher and logo
}\\

}
\endgroup}


\clearpage
\setcounter{page}{0}
\begin{document}
\titleGM
\section{Introduction}
We will start the \textit{Phase 4} report by a summary of our meeting with the client. We will then explain how the implementation of the requirements planned for this phase is going, and give information about the evolution of the architecture of our program.

\section{Meeting with client}

Last week, we had an appointment with the client, we will briefly explain here what was his feedback of the live version of our website. \\

We started off the meeting by asking a few questions that will allow us to improve the software.\\

We were wondering if it was important to keep tracks of the logins. It was indeed an important feature, as it allows to perform a faster search for a court. We also asked if it was important to display the prices of registration, which it indeed was. More generally, the software must have archives of all the actions performed, such as modifications of tournaments, users etc. This will be implemented during the final phase.\\

It was also necessary to have a smart registration, where the members of a pair fill their information and they're immediately assigned to a tournament. However, if a user doesn't match any tournament, there must be a message that clearly shows that the matching wasn't possible. We can also display a message to show which tournament a pair has been assigned to.\\

Lastly, as a court is assigned to a pool, we were told that the courts must be assigned according to how close it is to the headquarters. This must be generated by an algorithm and it could be manually changed.\\

We also received some feedback on other functionalities. He said it was a good idea to have a \textit{wiki}, since the website already uses one, and the home page was good. We could add some explanation about the tournaments but we must keep in mind that people usually come to the website only for registration and not to read the information the website.\\ 

The \textit{Court tab} isn't very clear, and the in the Tournament tab, we must not show publicly where people play in advance, because they must first come to the headquarters to pay for their registration.\\

Some more feedbacks : it is better to not display the final score of a match, putting the type of courts (already implemented), a staff member must be able to manually create a pool (when a tournament is created, a staff member can manually modify the tournament). More generally, we must make everything manually changeable.
\newpage
\section{Requirements, software description, Changes according to feedback}

The requirements below are the ones we planned to develop for the second phase. This time, every requirement has been implemented. \\
\subsection{For this phase}
	
\begin{itemize}
 
\item Files sharing between staff members (Eddy, Aurian)
\item Communication channel between staff members (Eddy)
\item The user can use a payment method among several options (Aurian)
\item The system must query AFT rankings (Optional)(See below)
\item The system can create smaller pools when it rains (Optional)(See below)
\\
\end{itemize}

It is important to note that the requirements haven't been implemented in this particular order.\\

You can see in the list above who implemented which requirement(s) for this phase. 

\subsection*{Implementation \& choices}

Maintenant dès que tous les scores de match d'une pool sont encodés, le gagnant de la pool s'affiche en vert et le knock-off tournament est créé dans la base de donnée.
si toutes les pools ont un gagnant

Allow a user to delete his email address from the DB

Maintenant on peut voir le bracket avec une jolie version graphique. Les joueurs gagnant des pools se retrouvent dans le bracket. Les scores pour l'instant ne sont éditable qu'à partir du admin (eddy va p-e s'en occuper)
\subsection*{Implementations according to meeting with client}

First of all, as requested, we implemented a system that allows to exchange messages. We also added a system that allows to share files that they can upload/download. The staff members already have other means to communicate, so these two features are important for very important announcements. They are available on a dedicated page. You can see in figure \ref{annonce} below how a staff member can put an announcement, and how one can upload files in figure \ref{file} \\
\begin{figure}[h]
  \caption{\label{annonce} Announcement creation}
  \includegraphics[scale=0.7]{annonce.png}
\end{figure}
\begin{figure}[h]
  \caption{\label{file} File sharing}
  \includegraphics[scale=0.7]{fichier.png}
\end{figure}


It is also now possible to modify the pools by taking into account the commentaries made.\\

We also added a functionality that allows to look for players and courts, but also pairs. We then added a list of all the players and all the courts.
Nous avons une fonction de recherche qui permet de retrouver des joueurs et des terrains, mais aussi des paires, on a aussi ajouté une liste de joueurs et de terrains. 

Secondly, we have implemented the smart registration. 
\subsubsection*{Bugs fixed}
In the last sprint, it was said in the previous report that the staff member interface had a different background to differenciate the navigation between a simple user and a staff user. We actually submitted a wrong live version where this feature was not implemented yet, which is now fixed.\\

One bigger issue was that we implemented a functionality that allowed to use an email so that a user doesn't have to fill all his information. By using the email, the system retrieved all the information from a past registration. However, this feature lead to information leak. From that point, we had to make a decision : we decided to send a confirmation email with a link, this link sends a token that allows the user to be recognized and to display his information. It is important to notice that in the case of a pair, only the information of the one who entered his email will be displayed, the other one will have to manually fill his form. It is a choice to have a more secured system, especially when two players are joined as a pair and if they're unknown to each other.\\

Some other bugs such as getting an error when trying to modify a user while being logged as a staff member, or one player couldn’t register alone leading to an error, or when logged as staff member when clicking on a tournament it changed to a user tournament. Those problems have all been fixed as well.

\subsection{The software}


Here is the live version of our website : http://sep2015e.herokuapp.com.  To get admin access you should add "/admin" in the url.\\

To get access to the data of our project, you can click on the link to our \textit{Github} repository : https://github.com/ivanahad/sep2015E\\
\subsubsection*{How to modify a tournament}

\subsubsection*{How }
\subsection{The software}


Here is the live version of our website : http://sep2015e.herokuapp.com.  To get admin access you should add "/admin" in the url.\\

To get access to the data of our project, you can click on the link to our \textit{Github} repository : https://github.com/ivanahad/sep2015E\\
\subsubsection*{How to modify a tournament}

\subsubsection*{How }

\section{For next phase}
\subsection{Tests to improve the software}
To improve our software, we are going to ask random people to test the website to see if it is user-friendly and easy to use, and then we will proceed to modify it accordingly.\\

We will also navigate randomly to search for bugs and strange behaviors that may arise after modifications of the database.\\
 
We will also test our website on different monitors with different screen sizes to see if it fits the window correctly.\\

\subsection{Documentation}
The documentation will be one of our main focuses. It will be divided into two parts: the first will be dedicated for the staff and admin and the second for the maintainer of the website. For the admin and staffs, the documentation will be a guide on how to navigate on the website and on how to basic stuffs like as modifying a pool, editing a player, etc. This documentation will be found on a wiki but links will be on the website to directly access the content specific to a part of the website. For the maintainer of the code, the documentation will be more technical. We will explain our implementaion, what we use and how we do it. 



\newpage
\section{Architecture discussion, choices \& UML Diagram}

You can see our UML diagram in Figure\ref{uml}. Since the last phase, we changed.\\

In the architecture, in this phase, some models such as the player and the tournament have been modified to fit the client's feedback. As we mentioned before, we added for a tournament two fields(age, gender) to describe the participants more accurately. (Note that the ORM must be updated but we felt that the modifications were trivial, as only two relationships would have been added, otherwise the model stays pretty much the same). 

\subsection*{Sequences diagram}

You can see in figure \ref{sequence} the sequence diagram for how a player registers to a tournament with an email confirmation. Note that this is a registration for a single player. For two players, it works the same way but it creates a pair instead. 

%AJOUTER SEQUENCE ICI ET ENLEVER LES %, AJUTER LE SCALE EN FONCTION
%\begin{figure}[position]
%   \caption{\label{sequence} Sequence diagram}
%  \includegraphics[scale=0.4]{sequence.png}
%\end{figure}

\section{Conclusion}

For this phase, we modified our software according to our meeting. We mainly focused on correcting bugs and making the website more user-friendly.

\end{document}