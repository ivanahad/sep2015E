\documentclass[a4paper, 12pt]{article}

\usepackage[french]{babel} 
\usepackage[utf8]{inputenc}
\usepackage[T1]{fontenc} 
\usepackage{amsmath}
\usepackage{amssymb}
\usepackage{listings}  
\usepackage{graphicx}
\usepackage[margin=2.5cm]{geometry}
\usepackage{amsmath,amsfonts,amssymb}
\lstset{
language=Java,
breaklines=true
}


\author{Aurian \bsc{De Potter}  \and Eddy \bsc{Ndizera}  \and Ivan \bsc{Ahad} \and Anthony \bsc{Dechamps} \and Ludovic \bsc{Fastré} \and Arnaud \bsc{Dethise} \and Jonathan \bsc{Legat}}

\title{INGI2255 - Software Engineering Project - Requirements, methodology and planning}

\date{\today}

\begin{document}

\maketitle
\section{Introduction}
This report contains all the information about how we are going to develop the software. First, we will talk about the  method and all the tools that we chose to use throughout the development. Secondly, there is the planning of all four phases of the project. Then comes our analysis of all the requirements that we were asked to put in the software. And finally, we present three use cases.  

\section{Important change}

During this phase we changed our implementation from mercurial to git. This was decided because it was more convenient. Firstly, all the members has had previous experience with git and secondly when it comes to managing collisions git is better. We also decide to use github instead of bitbucket. The advantage of bitbucket is that it has a private repository and manages mercurial. But as we are not using mercurial anymore, we are more than satisfied with GitHub because if the clients wish to publish the source code, than it is more interesting on gitHub due to it having a larger community.

\section{Requirements developed in this phase}
\begin{itemize}
\item The user can register as a pair to a tournament by filling in the form.
\item The user can register for activities during a tournament (barbecue, etc) and choose preferences such as taking responsibilities and payment method
\item Common page for staff members
\item The system can create pools and match ups
\item After a pool has been generated, a staff member can manually reorganize it 
\item An admin can close the registration for a tournament (and trigger the creation of pools)

\end{itemize}
\section{Conclusion}
As discussed in the report, we presented all the elements that  are required for the development of the software, and multiple information about this process, such as the method and tools chosen and the planning of the tasks to do for each intermediate deadlines. 

\end{document}