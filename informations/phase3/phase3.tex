
\documentclass[a4paper, 12pt]{article}

\usepackage[french]{babel} 
\usepackage[utf8]{inputenc}
\usepackage[T1]{fontenc} 
\usepackage{amsmath}
\usepackage[toc,page]{appendix}
\usepackage{amssymb}
\usepackage{listings}  
\usepackage{graphicx}
\usepackage[margin=2.5cm]{geometry}
\usepackage{amsmath,amsfonts,amssymb}
\usepackage{hyperref}
\lstset{
language=Java,
breaklines=true
}



\newcommand*{\plogo}{\fbox{$\mathcal{PL}$}} % Generic publisher logo

%----------------------------------------------------------------------------------------
%	TITLE PAGE
%----------------------------------------------------------------------------------------

\newcommand*{\titleGM}{\begingroup % Create the command for including the title page in the document
\hbox{ % Horizontal box
\hspace*{0.2\textwidth} % Whitespace to the left of the title page
\rule{2pt}{\textheight} % Vertical line
\hspace*{0.05\textwidth} % Whitespace between the vertical line and title page text
\parbox[b]{0.75\textwidth}{ % Paragraph box which restricts text to less than the width of the page

{\noindent\Huge\bfseries Software Project\\ Engineering }\\[2\baselineskip] % Title
{\Large \textit{Phase 3 Report}}\\[4\baselineskip] % Tagline or further description
{\Large \textbf{Project leader} : Hélène Verhaeghe}
\\
{\Large \textsc{\textbf{Group E}}\\\textsc{Aurian De Potter(Group leader)},\\ \textsc{Eddy Ndizera},\\ \textsc{Ivan Ahad},\\ \textsc{Arnaud Dethise},\\ \textsc{Ludovic Fastré},\\ \textsc{Anthony Dechamps},\\ \textsc{Geoffroy Husson},\\ \textsc{Jonathan Legat}} % Author name

\vspace{0.5\textheight} % Whitespace between the title block and the publisher
{\noindent \Large \textbf{INGI2255}}\\[\baselineskip] % Publisher and logo
}\\

}
\endgroup}


\clearpage
\setcounter{page}{0}
\begin{document}
\titleGM
\section{Introduction}
We will start the \textit{Phase 3} report by explaining how the implementation of the requirements planned for this phase is going. 

\section{Requirements developed in this phase, software description}

The requirements below are the ones we planned to develop for the second phase. This time, every requirement has been implemented. \\
\subsection{For this phase}
	
\begin{itemize}
 
\item The user can register on a tournament alone and be matched with another player (Arnaud)
\item An admin can start or close a tournament (Arnaud)
\item The user can leave a comment when registering (Arnaud)
\item A staff member can register a personal court as being available for the tournament (Already done in first phase)
\item A staff member can access the courts list (Arnaud and Geoffroy)
\item Users who registered their own court can see information about the players who will play on those (Ludovic)
\item A staff can manually assign a court for each match, or modify it in case of raining (Eddy)
\item A staff member can enter or edit match results data (Eddy and Arnaud)
\\
\end{itemize}
It is important to note that the requirements haven't been implemented in this particular order.\\

Many other features have also been implemented of course, to improve quality of the current website, such as the creation of a sponsor page, a contact page, a feature that allows a staff member to see and edit all the users information, etc. Moreover, we made sure that the website is \textit{mobile-friendly}, and some other improvements have been brought to the design.\\

You can see in the list above who implemented which requirement(s) for this phase. The other features were implemented by Ludovic, Aurian, Jonathan, Anthony, and Arnaud. Ivan took care of the report and made sure it respected all the criterion listed in the Criterion file. The team in general corrected the ORM diagram and updated the UML one for this phase. \\

Last sprint, the form for player registration was not working. This issue has now been fixed. Note that the players must choose a tournament to register for, otherwise the registration would be invalid. When this is done, the player will be redirected to the home page. The same mechanism goes for the court registration.  For future phases we planned on making it not possible for a player to register if there isn't any available tournament to register for. We also planned on adding a page showing that a registration has successfully been added.\\

\subsection{The software}

Here is the live version of our website : http://sep2015e.herokuapp.com.  To get admin access you should add "/admin" in the url. The username and the password are both \textbf{admin}. In the last phase our website only worked on our host. For this sprint, we made sure that it works even on the live version.\\

To get access to the data of our project, you can click on the link to our \textit{Github} repository : https://github.com/ivanahad/sep2015E\\ 
\section{Communication with the client \& Feedback}
Expliquer ici tout ce qui a changé en fonction de ce que le client avait demandé
 
\newpage
\section{Architecture discussion \& UML Diagram}

You can see our UML diagram in Figure\ref{uml2}. Since the last phase, we changed : expliquer tout ce qu'on a changé depuis la phase précédente, expliquer pourquoi on l'a fait comme ça.



\section{Conclusion}
Changment après le feedback du client

\begin{appendices}
\section*{Screenshots of the website}
\noindent

\end{appendices}
\end{document}