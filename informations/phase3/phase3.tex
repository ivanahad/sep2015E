
\documentclass[a4paper, 12pt]{article}

\usepackage[french]{babel} 
\usepackage[utf8]{inputenc}
\usepackage[T1]{fontenc} 
\usepackage{amsmath}
\usepackage[toc,page]{appendix}
\usepackage{amssymb}
\usepackage{listings}  
\usepackage{graphicx}
\usepackage[margin=2.5cm]{geometry}
\usepackage{amsmath,amsfonts,amssymb}
\usepackage{hyperref}
\lstset{
language=Java,
breaklines=true
}



\newcommand*{\plogo}{\fbox{$\mathcal{PL}$}} % Generic publisher logo

%----------------------------------------------------------------------------------------
%	TITLE PAGE
%----------------------------------------------------------------------------------------

\newcommand*{\titleGM}{\begingroup % Create the command for including the title page in the document
\hbox{ % Horizontal box
\hspace*{0.2\textwidth} % Whitespace to the left of the title page
\rule{2pt}{\textheight} % Vertical line
\hspace*{0.05\textwidth} % Whitespace between the vertical line and title page text
\parbox[b]{0.75\textwidth}{ % Paragraph box which restricts text to less than the width of the page

{\noindent\Huge\bfseries Software Project\\ Engineering }\\[2\baselineskip] % Title
{\Large \textit{Phase 3 Report}}\\[4\baselineskip] % Tagline or further description
{\Large \textbf{Project leader} : Hélène Verhaeghe}
\\
{\Large \textsc{\textbf{Group E}}\\\textsc{Aurian De Potter(Group leader)},\\ \textsc{Eddy Ndizera},\\ \textsc{Ivan Ahad},\\ \textsc{Arnaud Dethise},\\ \textsc{Ludovic Fastré},\\ \textsc{Anthony Dechamps},\\ \textsc{Geoffroy Husson},\\ \textsc{Jonathan Legat}} % Author name

\vspace{0.5\textheight} % Whitespace between the title block and the publisher
{\noindent \Large \textbf{INGI2255}}\\[\baselineskip] % Publisher and logo
}\\

}
\endgroup}


\clearpage
\setcounter{page}{0}
\begin{document}
\titleGM
\section{Introduction}
We will start the \textit{Phase 3} report by explaining how the implementation of the requirements planned for this phase is going. 

\section{Client's feedback}

In this section, we will briefly remind what were the change requests made from the client.\\

\subsection*{PLAYERS page}


 Pour les joueurs et les propriétaires de terrains cette partie du site semble très bien. Il manque cependant quelques éléments.

Page PLAYERS
Le type de tournoi dans lequel les joueurs jouent peut-être déterminé par leurs ages et leurs titres/sexes (deux champs à ajouter dans le formulaire d'inscription).
Je pense qu'il est plus judicieux de décomposer l'adresse en rue, numéro et boîte pour faciliter l'envoi du courrier par après.
Est-il possible d'ajouter ou supprimer des options comme le BBQ?
Je ne vois pas très bien à quoi correspond le champ "Activities".
Si l'inscription s’effectue ensemble, il est plus judicieux de faire le paiement pour la paire ensemble donc ne laisser qu'une seule case pour le choix du payement.
Il faudrait ajouter une vérification de la cohérence des champs avant l'envoi dans la base de donnée (Vérifier la forme du numéro de téléphone, que tous les champs soit complétés, envoyer un e-mail de confirmation)
\subsection*{OWNER page}
Page OWNERS
Tous les propriétaires sont des propriétaires privés donc il faudrait un formulaire avec leur adresse et l'adresse du terrain(pas toujours identique.
Une drop-down liste avec un choix de terrain plutôt qu'un champ libre avec les différents types de terrains (détaillés dans les slides je pense) serait mieux.
La possibilité de mettre une photo du terrain est une très bonne idée.
Nous ne souhaitons pas diffuser l'adresse des personnes participants au tournoi (qu'il soit propriétaire ou joueur), il n'est donc pas judicieux de proposer au public une recherche dans la base de donnée des propriétaires au bas de la page. Pour faire gagner du temps au propriétaire, un lien personnalisé pourrait leur être envoyé avec un formulaire pré-complété avec les informations fournies durant les années précédentes.

\subsection*{TOURNAMENTS page}
Page TOURNAMENTS
Il est possible pour n'importe qui de créer un tournoi sur votre site mais en réalité seul les membres du staff sont habilité à le faire. J'ai l'impression d'être tombé sur plusieurs versions de cette page, est-ce possible? (une version admin et une version pour le public)

\subsection*{SPONSORS page}
Page SPONSORS
Si possible pouvez vous faire une bannière identique sur chacune des pages destinées au public avec les sponsors?

La plupart des personnes qui participe au tournoi (joueur et propriétaire) sont francophone donc si vous avez le temps vous pouvez faire une version française de la partie destinée au public.

Courage pour la suite du travail, vous avez l'air sur la bonne voie. 
\section{Requirements, software description, Changes according to feedback}

The requirements below are the ones we planned to develop for the second phase. This time, every requirement has been implemented. \\
\subsection{For this phase}
	
\begin{itemize}
 
\item A staff member can visualize and print the evolution of the tour- nament in a fashionable and displayable form (both graphic and text), step by step, using the knockoff, and print the pool matchup sheet
\item An admin can create an admin account
\item An admin can modify his credentials
\item An admin can create a staff account
\item An admin can delete an account
\item The user can re-use old data linked to his email address to auto- matically fill a registration form
\item The system must confirm via email addresses
\item A staff member can use a mail list / newsletter functionality
\\
\end{itemize}
It is important to note that the requirements haven't been implemented in this particular order.\\

Many other features have also been implemented of course, to improve quality of the current website, such as the creation of a sponsor page, a contact page, a feature that allows a staff member to see and edit all the users information, etc. Moreover, we made sure that the website is \textit{mobile-friendly}, and some other improvements have been brought to the design.\\

You can see in the list above who implemented which requirement(s) for this phase. The other features were implemented by Ludovic, Aurian, Jonathan, Anthony, and Arnaud. Ivan took care of the report and made sure it respected all the criterion listed in the Criterion file. The team in general corrected the ORM diagram and updated the UML one for this phase. \\

Last sprint, the form for player registration was not working. This issue has now been fixed. Note that the players must choose a tournament to register for, otherwise the registration would be invalid. When this is done, the player will be redirected to the home page. The same mechanism goes for the court registration.  For future phases we planned on making it not possible for a player to register if there isn't any available tournament to register for. We also planned on adding a page showing that a registration has successfully been added.\\

\subsection{The software}

Here is the live version of our website : http://sep2015e.herokuapp.com.  To get admin access you should add "/admin" in the url. The username and the password are both \textbf{admin}. In the last phase our website only worked on our host. For this sprint, we made sure that it works even on the live version.\\

To get access to the data of our project, you can click on the link to our \textit{Github} repository : https://github.com/ivanahad/sep2015E\\ 
\section{Architecture discussion \& UML Diagram}

You can see our UML diagram in Figure\ref{uml}. Since the last phase, we changed : expliquer tout ce qu'on a changé depuis la phase précédente, expliquer pourquoi on l'a fait comme ça.



\section{Conclusion}
Changment après le feedback du client, on a implémenté tous les req, les diagram? 

\begin{appendices}
\noindent

\end{appendices}
\end{document}