
\documentclass[a4paper, 12pt]{article}

\usepackage[french]{babel} 
\usepackage[utf8]{inputenc}
\usepackage[T1]{fontenc} 
\usepackage{amsmath}
\usepackage[toc,page]{appendix}
\usepackage{amssymb}
\usepackage{listings}  
\usepackage{graphicx}
\usepackage[margin=2.5cm]{geometry}
\usepackage{amsmath,amsfonts,amssymb}
\usepackage{hyperref}
\lstset{
language=Java,
breaklines=true
}



\newcommand*{\plogo}{\fbox{$\mathcal{PL}$}} % Generic publisher logo

%----------------------------------------------------------------------------------------
%	TITLE PAGE
%----------------------------------------------------------------------------------------

\newcommand*{\titleGM}{\begingroup % Create the command for including the title page in the document
\hbox{ % Horizontal box
\hspace*{0.2\textwidth} % Whitespace to the left of the title page
\rule{2pt}{\textheight} % Vertical line
\hspace*{0.05\textwidth} % Whitespace between the vertical line and title page text
\parbox[b]{0.75\textwidth}{ % Paragraph box which restricts text to less than the width of the page

{\noindent\Huge\bfseries Software Project\\ Engineering }\\[2\baselineskip] % Title
{\Large \textit{Phase 3 Report}}\\[4\baselineskip] % Tagline or further description
{\Large \textbf{Project leader} : Hélène Verhaeghe}
\\
{\Large \textsc{\textbf{Group E}}\\\textsc{Aurian De Potter(Group leader)},\\ \textsc{Eddy Ndizera},\\ \textsc{Ivan Ahad},\\ \textsc{Arnaud Dethise},\\ \textsc{Ludovic Fastré},\\ \textsc{Anthony Dechamps},\\ \textsc{Geoffroy Husson},\\ \textsc{Jonathan Legat}} % Author name

\vspace{0.5\textheight} % Whitespace between the title block and the publisher
{\noindent \Large \textbf{INGI2255}}\\[\baselineskip] % Publisher and logo
}\\

}
\endgroup}


\clearpage
\setcounter{page}{0}
\begin{document}
\titleGM
\section{Introduction}
We will start the \textit{Phase 3} report by listing all the important elements that you wanted us to modify. We will then explain how the implementation of the requirements planned for this phase is going, and give information about the evolution of the architecture of our program.

\section{Client's feedback}

In this section, we will briefly list what were the change requests that you made.\\

The players and owner pages were said to fit correctly, with some details that were missing.
\subsection*{PLAYERS page}

The type of tournament in which the players play can be determined by their age and their situation/genre, those two fields are to be added to the form. You thought it might be a better approach to break down the address into the street name, the number, and the mailbox number. You also asked if we could make it possible to add/delete some options such as the barbecue, to combine the checkout if two players register together, and to add a verification system to check if all the required fields have been filled by the mean of a confirmation email.\\

\subsection*{OWNER page}

Since the owners are private owners, there should be a form with their address and the court's address which isn't always the same. The courts list should be presented as a drop-down list instead of an empty field.\\

The possibility of putting a picture of the court was said to be a good idea.\\

As you were not willing to diffuse the participants' addresses, it wouldn't be a good idea to allow users to make a search in the database of owners. Morever, a customed linked could be used with a pre-completed form with all the information from previous years to make it quicker for owners to re-register their court.

\subsection*{TOURNAMENTS page}
It shouldn't be possible for anybody to create a tournament, this feature should be only allowed for staff members. This page also seems to have several versions.

\subsection*{SPONSORS page}

You asked us to make an identical banner for every public page with the sponsors. Also, as the website is mostly used by french users, the public part should rather be in french.

 
\section{Requirements, software description, Changes according to feedback}

The requirements below are the ones we planned to develop for the second phase. This time, every requirement has been implemented. \\
\subsection{For this phase}
	
\begin{itemize}
 
\item A staff member can visualize and print the evolution of the tour- nament in a fashionable and displayable form (both graphic and text), step by step, using the knockoff, and print the pool matchup sheet
\item An admin can create an admin account
\item An admin can modify his credentials
\item An admin can create a staff account
\item An admin can delete an account
\item The user can re-use old data linked to his email address to auto- matically fill a registration form
\item The system must confirm via email addresses
\item A staff member can use a mail list / newsletter functionality
\\
\end{itemize}
It is important to note that the requirements haven't been implemented in this particular order.\\

Many other features have also been implemented of course, to improve quality of the current website, such as the creation of a sponsor page, a contact page, a feature that allows a staff member to see and edit all the users information, etc. Moreover, we made sure that the website is \textit{mobile-friendly}, and some other improvements have been brought to the design.\\

You can see in the list above who implemented which requirement(s) for this phase. The other features were implemented by Ludovic, Aurian, Jonathan, Anthony, and Arnaud. Ivan took care of the report and made sure it respected all the criterion listed in the Criterion file. The team in general corrected the ORM diagram and updated the UML one for this phase. \\

Last sprint, the form for player registration was not working. This issue has now been fixed. Note that the players must choose a tournament to register for, otherwise the registration would be invalid. When this is done, the player will be redirected to the home page. The same mechanism goes for the court registration.  For future phases we planned on making it not possible for a player to register if there isn't any available tournament to register for. We also planned on adding a page showing that a registration has successfully been added.\\

\subsection{The software}

Here is the live version of our website : http://sep2015e.herokuapp.com.  To get admin access you should add "/admin" in the url. The username and the password are both \textbf{admin}. In the last phase our website only worked on our host. For this sprint, we made sure that it works even on the live version.\\

To get access to the data of our project, you can click on the link to our \textit{Github} repository : https://github.com/ivanahad/sep2015E\\ 
\section{Architecture discussion \& UML Diagram}

You can see our UML diagram in Figure\ref{uml}. Since the last phase, we changed : expliquer tout ce qu'on a changé depuis la phase précédente, expliquer pourquoi on l'a fait comme ça.



\section{Conclusion}
Changment après le feedback du client, on a implémenté tous les req, les diagram? 

\begin{appendices}
\noindent

\end{appendices}
\end{document}