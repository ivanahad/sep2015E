\documentclass[a4paper, 12pt]{article}

\usepackage[french]{babel} 
\usepackage[utf8]{inputenc}
\usepackage[T1]{fontenc} 
\usepackage{amsmath}
\usepackage{amssymb}
\usepackage{listings}  
\usepackage{graphicx}
\usepackage[margin=2.5cm]{geometry}
\usepackage{amsmath,amsfonts,amssymb}
\usepackage{hyperref}
\lstset{
language=Java,
breaklines=true
}



\newcommand*{\plogo}{\fbox{$\mathcal{PL}$}} % Generic publisher logo

%----------------------------------------------------------------------------------------
%	TITLE PAGE
%----------------------------------------------------------------------------------------

\newcommand*{\titleGM}{\begingroup % Create the command for including the title page in the document
\hbox{ % Horizontal box
\hspace*{0.2\textwidth} % Whitespace to the left of the title page
\rule{2pt}{\textheight} % Vertical line
\hspace*{0.05\textwidth} % Whitespace between the vertical line and title page text
\parbox[b]{0.75\textwidth}{ % Paragraph box which restricts text to less than the width of the page

{\noindent\Huge\bfseries Software Project\\ Engineering }\\[2\baselineskip] % Title
{\Large \textit{Phase 2 Report}}\\[4\baselineskip] % Tagline or further description
{\Large \textbf{Project leader} : Hélène Verhaeghe}
\\
{\Large \textsc{\textbf{Group E}}\\\textsc{Aurian De Potter(Group leader)},\\ \textsc{Eddy Ndizera},\\ \textsc{Ivan Ahad},\\ \textsc{Arnaud Dethise},\\ \textsc{Ludovic Fastré},\\ \textsc{Anthony Dechamps},\\ \textsc{Geoffroy Husson},\\ \textsc{Jonathan Legat}} % Author name

\vspace{0.5\textheight} % Whitespace between the title block and the publisher
{\noindent \Large \textbf{INGI2255}}\\[\baselineskip] % Publisher and logo
}\\

}
\endgroup}


\clearpage
\setcounter{page}{0}
\begin{document}
\titleGM
\section{Introduction}


\section{Important change}
N'a-t-on pas implémenté qqchose ? expliquer si oui

\section{Requirements developed in this phase}

	
	
\begin{itemize}
 
\item ahaiehgihergoiherg
\end{itemize}

Qui a fait quoi
\section{ORM diagram CORRECTED}

Expliquer ce qu'on a changé
 Fig.\ref{ORM_diagram} shows our ORM diagram. It gives an idea about how the different objects used by the website are linked together. We will give a brief description of what the object represents and its relations to others. We will ask you to zoom in the PDF file, as it was a challenge to make the diagrams fit completely in the report, we apologize for the inconvenience.\\
 
 The \textit{user} represents a player participating in the tournament. It is characterized by a name, an email, an address and the level of the player (meaning how good he is at tennis). \\
 
 A \textit{pair} is formed of two users. A single pair can only participate in one \textit{pool} of the \textit{tournament}. A pool contains pairs and their \textit{match} (matchups between pairs). A match is characterized by the pairs confronting each other and the \textit{court} in which they will play. The court object contains information about the court owner, the address of it and finally some comments about it like the type of surface. \\
 
 As we said, a tournament is composed of pools but also of a knock-off tournament that we called \textit{tree} in our diagram. Moreover, a tournament is defined by its \textit{category}. Finally, the tree is composed of \textit{nodes} that is basically a pair of two users. \\
 
 This concludes the description about the ORM diagram. We still must inform you the diagram may change. We may add more objects or change the links between them. But we will take care as to write out the changes done in case we modify it.
 


\newpage
\section{UML Diagram for this phase}


\section{Conclusion}

\end{document}